\documentclass[12pt]{article}

\usepackage[margin=1in]{geometry}
\usepackage{amsmath,amsthm,amssymb}
\usepackage{float}
\usepackage{graphicx}

\newcommand{\N}{\mathbb{N}}
\newcommand{\Z}{\mathbb{Z}}
\newcommand{\abs}[1]{\left| #1 \right|}
\newcommand{\ceil}[1]{\left\lceil #1 \right\rceil}
\newcommand{\floor}[1]{\left\lfloor #1 \right\rfloor}

\newenvironment{theorem}[2][Theorem]{\begin{trivlist}
\item[\hskip \labelsep {\bfseries #1}\hskip \labelsep {\bfseries #2.}]}{\end{trivlist}}
\newenvironment{lemma}[2][Lemma]{\begin{trivlist}
\item[\hskip \labelsep {\bfseries #1}\hskip \labelsep {\bfseries #2.}]}{\end{trivlist}}
\newenvironment{exercise}[2][Exercise]{\begin{trivlist}
\item[\hskip \labelsep {\bfseries #1}\hskip \labelsep {\bfseries #2.}]}{\end{trivlist}}
\newenvironment{problem}[2][Problem]{\begin{trivlist}
\item[\hskip \labelsep {\bfseries #1}\hskip \labelsep {\bfseries #2.}]}{\end{trivlist}}
\newenvironment{question}[2][Question]{\begin{trivlist}
\item[\hskip \labelsep {\bfseries #1}\hskip \labelsep {\bfseries #2.}]}{\end{trivlist}}
\newenvironment{corollary}[2][Corollary]{\begin{trivlist}
\item[\hskip \labelsep {\bfseries #1}\hskip \labelsep {\bfseries #2.}]}{\end{trivlist}}

\DeclareMathOperator{\sech}{sech}
\DeclareMathOperator{\csch}{csch}

\begin{document}

\title{CS6210: Homework 2}
\author{Christopher Mertin}
\date{September 15, 2016}
\maketitle

\begin{enumerate}
%%%%%% Problem 1 %%%%%%
\item Consider the fixed point iteration $x_{k+1} = g\left( x_{k}\right),\ k=\{0,1,\ldots\}$
and let all the assumptions of the Fixed Point Theorem hold. Use a Taylor's series expansion
to show that the order of convergence depends on how many of the derivatives of $g$ vanish
at $x = x^{*}$. Use your result to state how fast (at least) a fixed point iteration
is expected to converge if $g^{\prime} = \cdots = g^{(r)}\left( x^{*}\right) = 0$,
where the integer $r \geq 1$ is given.

{\bf Solution:}

The definition of the Taylor Series Expansion for a function $g(x)$ around $x = x^{*}$ is defined as:

\begin{align*}
g(x) &= \sum_{n=0}^{\infty} \frac{g^{(n)}\left( x^{*}\right)}{n!} \left( x - x^{*}\right)^{n}\\
g\left( x_{k}\right) &= \sum_{n=0}^{\infty} \frac{g^{(n)}\left( x^{*}\right)}{n!} \left( x_{k} - x^{*}\right)^{n}\\
x_{k+1} - x^{*} &= \xi_{k+1} = \sum_{n=1}^{\infty} \frac{g^{(n)}\left( x^{*}\right)}{n!}\xi_{k}^{n}
\end{align*}

%%%%%% Problem 2 %%%%%%
\item Consider the function $g(x) = x^{2} + \frac{3}{16}$
\begin{enumerate}
  \item This function has two fixed points. What are they?
  \item Consider the fixed point iteration $x_{k+1} = g\left( x_{k}\right)$ for this $g$.
  For which of the points you have found in $(a)$ can you be sure that the iterations
  will converge to that fixed point? Briefly justify your answer. You may assume that
  the initial guess is sufficiently close to the fixed point.
  \item For the point or points you found in $(b)$, roughly how many iterations will be
  required to reduce the convergence error by a factor of 10?
\end{enumerate}

%%%%%% Problem 3 %%%%%%
\item It is known that the order of convergence of the secant method is $p = \frac{1 + \sqrt{5}}{2} \approx 1.618\ldots$
and that of Newton's method is $p = 2$. Suppose that evaluating $f^{\prime}$ costs approximately
$\alpha$ times the cost of approximating $f$. Determine approximately for what values of $\alpha$
Newton's method is more efficient (in terms of number of function evaluations) than the secant method.
You may neglect the asymptotic error constants in your calculations. Assume that both methods
are starting with initial guesses of a similar quality.

%%%%%% Problem 4 %%%%%%
\item The function

\[
f(x) = (x-1)^{2}e^{x}
\]

has a double root at $x = 1$.

\begin{enumerate}
\item Derive Newton's iteration for this function. Show that the iteration is well-defined
so long as $x_{k} \neq -1$ and that the convergence rate is expected to be similar to that
of the bisection method (and certainly not quadratic).
\item Implement Newton's method and observe its performance starting from $x_{0} = 2$.
\item How easy would it be to apply the bisection method? Explain.
\end{enumerate}

%%%%%% Problem 5 %%%%%%
\item Given $a > 0$, we wish to compute $x = \ln (a)$ using addition, subtraction, multiplication,
division, and the exponential function $e^{x}$.
\begin{enumerate}
\item Suggest an iterative formula based on Newton's method, and write it in a way suitable
for numerical computation
\item Show that your formula converges quadratically
\item Write down an iterative formula based on the secant method
\item State which of the secant and Newton's methods is expected to perform better in
this case in terms of overall number of exponential function evaluations. Assume a fair
comparison, {\em i.e.} same floating point system, ``same quality'' initial guesses, and idential
convergence criterion.
\end{enumerate}

%%%%%% Problem 6 %%%%%%
\item For $x>0$ consider the equation

\[
x + \ln(x) = 0
\]

It is a reformulation of the equation of Example 3.4

\begin{enumerate}
  \item Show analytically that there is exactly one root, $0 < x^{*} < \infty$
  \item Plot a graph of the function on the interval $[0.1, 1]$
  \item As you can see from the graph, the root is between $0.5$ and $0.6$. Write
  {\sc Matlab} routines for finding the root, using the following:
  \begin{enumerate}
    \item The bisection method, with the initial interval $[0.5,0.6]$. Explain
    why this choice of the intial interval is valid.
    \item A linearly convergent fixed point interation, with $x_{0} = 0.5$. Show that
    the conditions of the Fixed Point Theorem (for the function $g$ you have selected)
    are satisfied.
    \item Newton's method, with $x_{0} = 0.5$.
    \item The secant method, with $x_{0} = 0.5$ and $x_{1} = 0.6$.
  \end{enumerate}
  For each of the methods:
  \begin{itemize}
    \item Use $\abs{ x_{k} - x_{k-1}} < 10^{-10}$ as convergence criterion
    \item Print out eh iterates and show the progress in the number of correct decimal
    digits throughout the iteration.
    \item Explain the convergence behavior and how it matches theoretical expectations
  \end{itemize}
\end{enumerate}
\end{enumerate}

%\begin{proof}
%Blah, blah, blah.  Here is an example of the \texttt{align} environment:
%Note 1: The * tells LaTeX not to number the lines.  If you remove the *, be sure to remove it below, too.
%Note 2: Inside the align environment, you do not want to use $-signs.  The reason for this is that this is already a math environment. This is why we have to include \text{} around any text inside the align environment.
%\begin{align*}
%\sum_{i=1}^{k+1}i & = \left(\sum_{i=1}^{k}i\right) +(k+1)\\
%& = \frac{k(k+1)}{2}+k+1 & (\text{by inductive hypothesis})\\
%& = \frac{k(k+1)+2(k+1)}{2}\\
%& = \frac{(k+1)(k+2)}{2}\\
%& = \frac{(k+1)((k+1)+1)}{2}.
%\end{align*}
%\end{proof}

\end{document}
