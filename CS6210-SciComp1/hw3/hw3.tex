\documentclass[12pt]{article}

\usepackage[margin=1in]{geometry}
\usepackage{amsmath,amsthm,amssymb}

\newcommand{\N}{\mathbb{N}}
\newcommand{\Z}{\mathbb{Z}}

\newenvironment{theorem}[2][Theorem]{\begin{trivlist}
\item[\hskip \labelsep {\bfseries #1}\hskip \labelsep {\bfseries #2.}]}{\end{trivlist}}
\newenvironment{lemma}[2][Lemma]{\begin{trivlist}
\item[\hskip \labelsep {\bfseries #1}\hskip \labelsep {\bfseries #2.}]}{\end{trivlist}}
\newenvironment{exercise}[2][Exercise]{\begin{trivlist}
\item[\hskip \labelsep {\bfseries #1}\hskip \labelsep {\bfseries #2.}]}{\end{trivlist}}
\newenvironment{problem}[2][Problem]{\begin{trivlist}
\item[\hskip \labelsep {\bfseries #1}\hskip \labelsep {\bfseries #2.}]}{\end{trivlist}}
\newenvironment{question}[2][Question]{\begin{trivlist}
\item[\hskip \labelsep {\bfseries #1}\hskip \labelsep {\bfseries #2.}]}{\end{trivlist}}
\newenvironment{corollary}[2][Corollary]{\begin{trivlist}
\item[\hskip \labelsep {\bfseries #1}\hskip \labelsep {\bfseries #2.}]}{\end{trivlist}}


\DeclareMathOperator{\sech}{sech}
\DeclareMathOperator{\csch}{csch}

\begin{document}

\title{CS6210: Homework 1}
\author{Christopher Mertin}
\date{September 6, 2016}
\maketitle

\begin{enumerate}
%%%%%% Problem 1 %%%%%%
\item Carry out calculations similar to those of Example 1.3 for approximating
the derivative of the function $f(x)=e^{-2x}$ evaluated at $x_{0}=0.5$. Observe
similarities and differences by comparing your graph against that in Figure 1.3.

%%%%%% Problem 2 %%%%%%
\item Following Example 1.5, assess the conditioning of the problem of evaluating
\[
  g(x)=\tanh{(cx)} = \frac{e^{cx}-e^{-cx}}{e^{cx}+e^{-cx}}
\]

near $x=0$ as the positive parameter $c$ grows.

{\bf Solution:}

The condition number measures how sensitive the output of a function is to small
changes in the input. It can be calculated by:

\begin{align*}
\text{Condition Number}(g) &= x\frac{g^{\prime}(x)}{g(x)}
\end{align*}

which for the instance of this problem gives us

\begin{align*}
  g(x) &= \frac{e^{cx} - e^{-cx}}{e^{cx}+e^{-cx}} = \frac{\frac{e^{cx}-e^{-cx}}{2}}{\frac{e^{cx}+e^{-cx}}{2}}\\
        &= \frac{\sinh(x)}{\cosh(x)}\\
  g^{\prime}(x) &= c \sech^{2}(cx)\\
                &= \frac{c}{\cosh^{2}(cx)}
\end{align*}

with using these formulas, we can plug it into the definition of the condition
number we get

\begin{align*}
\text{Condition Number}(g) &= x\frac{g^{\prime}(x)}{g(x)} = \frac{cx}{\cosh^{2}(x)\cdot \tanh(cx)}\\
                            &= cx\frac{\cosh(cx)}{\sinh(cx)} = cx\cdot \coth(cx)
\intertext{Where we can take the limit of the condition number as $x$ approaches $0$, giving}
\lim_{x\rightarrow 0} cx\cdot \coth(cx) &= 1
\end{align*}

This shows that the funcion's sensitivity is not dependent upon $c$ and is well conditioned.


%%%%%% Problem 3 %%%%%%
\item The function $f_{1}\left(x_{0},h\right) = \sin\left(x_{0}+h\right) - \sin\left(x_{0}\right)$
can be transformed into another form, $f_{2}\left(x_{0},h\right)$, using the
trigonometric formula
\[
\sin(\phi) - \sin(\psi) = 2\cos\left(\frac{\phi + \psi}{2}\right)\sin\left(\frac{\phi - \psi}{2}\right)
\]
Thus $f_{1}$ and $f_{2}$ have the same values, in exact arithmetic, for any given
argument values $x_{0}$ and $h$.
\begin{enumerate}
\item Derive $f_{2}\left(x_{0},h\right)$
\item Suggest a formula that avoids cancellation errors for computing the approximation
$\left(\left( f\left( x_{0}+h\right) - f\left(x_{0}\right)\right) /h\right)$
to the derivative of $f(x)=\sin(x)$ at $x=x_{0}$. Write a {\sc Matlab} program
that implements your formula and computes an approximation of $f^{\prime}(1.2)$
for $h = \left\{ 10^{-20},10^{-19},\ldots,1\right\}$.
\item Explain the difference in accuracy between your results and the results
reported in Example 1.3.
\end{enumerate}

%%%%%% Problem 4 %%%%%%
\item The function $f_{1}(x,\delta) = \cos(x+\delta)-\cos(x)$ can be transformed
into another form, $f_{2}(x,\delta)$, using the trigonometric formula
\[
\cos(\phi) - \cos(\psi) = -2\sin\left(\frac{\phi + \psi}{2}\right)\sin\left(\frac{\phi - \psi}{2}\right)
\]
Thus, $f_{1}$ and $f_{2}$ have the same values, in exact arithmetic, for any given
argument values $x$ and $\delta$.
\begin{enumerate}
\item Show that, analytically, $f_{1}(x,\delta)/\delta$ or $f_{2}(x,\delta)/\delta$
are effective approximations of the function $(-\sin(x))$ for $\delta$ sufficiently small.
\item Derive $f_{2}(x,\delta)$
\item Write a {\sc Matlab} script which will calculate $g_{1}(x,\delta) = f_{1}(x,\delta)/\delta + \sin(x)$
and $g_{2}(x,\delta) = f_{2}(x,\delta)/\delta + \sin(x)$ for $x = 3$ and $\delta = 10^{-11}$.
\item Explain the difference in the results of the two calculations.
\end{enumerate}

%%%%%% Problem 5 %%%%%%
\item Consider the approximation of the first derivative
\[
f^{\prime}(x) \approx \frac{f(x+h)-f(x)}{h}
\]
The {\em truncation error} for this formula is $\mathcal{O}(h)$. Suppose that the
absolute error in evaluating the function $f$ is bounded by $\epsilon$ and
let us ignore the errors generated in basic arithmetic operations.
\begin{enumerate}
\item Show that the total computational error (truncation error and rounding combined)
is bounded by
\[
\frac{Mh}{2} + \frac{2\epsilon}{h}
\]
where $M$ is a bound on $\left| f^{\prime\prime}(x)\right|$.
\item What is the value of $h$ for which the above is minimized?
\item The rounding unit we employ is approximately equal to $10^{-16}$. Use this
to explain the behavior of the graph in Example 1.3. Make sure to explain the shape
of the graph as well as the value where the apparent minimum is attained.
\item It is not difficult to show, using Taylor expansions, that $f^{\prime}(x)$ can
be approximated more accurately (in terms of truncation error) by
\[
f^{\prime}(x) \approx \frac{f(x+h) - f(x-h)}{2h}
\]
For this approximation, the truncation error is $\mathcal{O}\left(h^{2}\right)$.
Generate a graph similar to Figure 1.3 for the same function and the same value
of $x$, namely, for $\sin(1.2)$, and compare the two graphs. Explain the meaning
of your results.
\end{enumerate}

%%%%%% Problem 6 %%%%%%
\item In the statistical treatment of data one often needs to compute the quantities
\[
\bar{x} = \frac{1}{n}\sum_{i=1}^{n}x_{i},\quad \sigma^{2}=\frac{1}{n}\sum_{i=1}^{n}\left(x_{i}-\bar{x}\right)^{2}
\]
where $\left\{ x_{1}, x_{2}, \ldots, x_{n}\right\}$ are the given data. Assume that $n$
is large, say, $n = 10000$. It is easy to see that $\sigma^{2}$ can also be written as
\[
\sigma^{2} = \frac{1}{n}\sum_{i=1}^{n} x_{i}^{2} - \bar{x}^{2}
\]
\begin{enumerate}
\item Which of the two methods to calculate $\sigma^{2}$ is cheaper in terms of
overall computational cost? Assume $\bar{x}$ has already been calculated and give
the operation counts for these two options.
\item Which of the two methods is expected to give more accurate results for
$\sigma^{2}$ in general?
\item Give a small example, using a decimal system with precision $t = 2$ and
numbers of your choice, validate your claims.
\end{enumerate}
\end{enumerate}

%\begin{proof}
%Blah, blah, blah.  Here is an example of the \texttt{align} environment:
%Note 1: The * tells LaTeX not to number the lines.  If you remove the *, be sure to remove it below, too.
%Note 2: Inside the align environment, you do not want to use $-signs.  The reason for this is that this is already a math environment. This is why we have to include \text{} around any text inside the align environment.
%\begin{align*}
%\sum_{i=1}^{k+1}i & = \left(\sum_{i=1}^{k}i\right) +(k+1)\\
%& = \frac{k(k+1)}{2}+k+1 & (\text{by inductive hypothesis})\\
%& = \frac{k(k+1)+2(k+1)}{2}\\
%& = \frac{(k+1)(k+2)}{2}\\
%& = \frac{(k+1)((k+1)+1)}{2}.
%\end{align*}
%\end{proof}

\end{document}
