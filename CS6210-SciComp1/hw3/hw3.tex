\documentclass[12pt]{article}

\usepackage[margin=1in]{geometry}
\usepackage{amsmath,amsthm,amssymb}
\usepackage{float}
\usepackage{graphicx}
\usepackage{bbold}

\newcommand{\N}{\mathbb{N}}
\newcommand{\Z}{\mathbb{Z}}
\newcommand{\abs}[1]{\left| #1 \right|}
\newcommand{\ceil}[1]{\left\lceil #1 \right\rceil}
\newcommand{\floor}[1]{\left\lfloor #1 \right\rfloor}
\newcommand{\pprime}{\prime \prime}
\newcommand{\BigO}[1]{\mathcal{O}\left( #1 \right)}
\newcommand{\Id}{\mathbb{1}}

\newenvironment{theorem}[2][Theorem]{\begin{trivlist}
\item[\hskip \labelsep {\bfseries #1}\hskip \labelsep {\bfseries #2.}]}{\end{trivlist}}
\newenvironment{lemma}[2][Lemma]{\begin{trivlist}
\item[\hskip \labelsep {\bfseries #1}\hskip \labelsep {\bfseries #2.}]}{\end{trivlist}}
\newenvironment{exercise}[2][Exercise]{\begin{trivlist}
\item[\hskip \labelsep {\bfseries #1}\hskip \labelsep {\bfseries #2.}]}{\end{trivlist}}
\newenvironment{problem}[2][Problem]{\begin{trivlist}
\item[\hskip \labelsep {\bfseries #1}\hskip \labelsep {\bfseries #2.}]}{\end{trivlist}}
\newenvironment{question}[2][Question]{\begin{trivlist}
\item[\hskip \labelsep {\bfseries #1}\hskip \labelsep {\bfseries #2.}]}{\end{trivlist}}
\newenvironment{corollary}[2][Corollary]{\begin{trivlist}
\item[\hskip \labelsep {\bfseries #1}\hskip \labelsep {\bfseries #2.}]}{\end{trivlist}}

\DeclareMathOperator{\sech}{sech}
\DeclareMathOperator{\csch}{csch}

\begin{document}

\title{CS6210: Homework 3}
\author{Christopher Mertin}
\date{October 4, 2016}
\maketitle

\begin{enumerate}
%%%%%% Problem 1 %%%%%%
\item The {\em condition number} of an eigenvalue $\lambda$ of a given matrix $A$ is defined as

\[
s(\lambda) = \frac{1}{\mathbf{x}^{T}\mathbf{w}}
\]

where $\mathbf{x}$ is a (right) eigenvector of the matrix, satisfying $A\mathbf{x} = \lambda \mathbf{x}$,
and $\mathbf{w}$ is a left eigenvector, satisfying $\mathbf{w}^{T}A = \lambda \mathbf{w}^{T}$.
Both $\mathbf{x}$ and $\mathbf{w}$ are assumed to have a unit $\ell_{2}$-norm. Loosly speaking,
the condition number determines the difficulty of computing the eigenvalue in question accurately;
the smaller $s(\lambda)$ is, the more numerically stable the computation is expected to be.

Determine the condition number of the eigenvalue 4 for the two matrices discussed in Example 4.7.
Explain the meaning of your results and how they are related to the observations made in the example.

{\bf Solution:}

%%%%%% Problem 2 %%%%%%
\item The {\em Gauss-Jordan method} used to solve the prototype linear system can be described
as follows. Augment $A$ by the right-hand-side vector $\mathbf{b}$ and proceed as in Gaussian Elimination,
except use the pivot element $a_{k,k}^{(k-1)}$ to eliminate not only $a_{i,k}^{(k-1)}$ for $i = \{ k+1, \ldots, n\}$
but also the elements $a_{i,k}^{(k-1)}$ for $i = \{1, \ldots, k-1\}$, {\em i.e.}, all elements in the $k^{th}$ column
other than the pivot. Upon reducing $(A|\mathbf{b})$ into

\[
\left[
\begin{array}{cccc|c}
  a_{1,1}^{(n-1)} & 0 & \cdots & 0 & b_{1}^{(n-1)}\\
  0 & a_{2,2}^{(n-1)} & \ddots & \vdots & b_{2}^{(n-1)}\\
  \vdots & \ddots & \ddots & 0 & \vdots\\
  0 & \cdots & 0 & a_{n,n}^{(n-1)} & b_{n}^{(n-1)}
\end{array}\right]
\]

the solution is obtained by setting

\[
x_{k} = \frac{b_{k}^{(n-1)}}{a_{k,k}^{(n-1)}},\quad k = \{ 1, \ldots, n\}
\]

This procedure circumvents the backward substitution part necessary for the Gaussian Elimination algorithm.

\begin{enumerate}
\item Write a pseudocode for this Gauss-Jordan procedure using, {\em e.g.}, the same format
as for the one appearing in Section 5.2 for Gaussian Elimination. You may assume that no
pivoting ({\em i.e.}, no row interchanging) is required.

{\bf Solution:}

\item Show that the Gauss-Jordan method requires $n^{3} + \BigO{n^{2}}$ floating point operations
for one right-hand-side vector $\mathbf{b}$ -- roughly 50\% more than what's needed for Gaussian Elimination

{\bf Solution:}


\end{enumerate}

%%%%%% Problem 3 %%%%%%
\item Let $A$ and $T$ be two nonsingular $n \times n$ real matrices. Furthermore, suppose we are given
two matrices $L$ and $U$ such that $L$ is the unit lower triangular, $U$ is the upper triangular, and

\[
TA = LU
\]

Write an algorithm that will solve the problem

\[
A\mathbf{x} = \mathbf{b}
\]

for any given vector $\mathbf{b}$ in $\BigO{n^{2}}$ complexity. First, explain briefly yet clearly
why your algorithm requires only $\BigO{n^{2}}$ flops (you may assume without proof that solving an upper triangular
or a lower triangular system requires only $\BigO{n^{2}}$ flops). Then, specify your alogirhtm in detail
(including the details for lower and upper triangular systems) using pseudocode or a {\sc Matlab} script.

{\bf Solution:}


%%%%%% Problem 4 %%%%%%
\item The classical way to invert a matrix $A$ in a basic linear algebra course augments $A$ by the $n\times n$
identity matrix $\Id$ and applies the Gauss-Jordan algorithm of Exercise 2 to this augmented matrix
(including the solutin part, {\em i.e.}, the division by the pivots $a_{k,k}^{(n-1)}$). Then $A^{-1}$
shows up where $\Id$ initially was.

How many floating point operations are required for this method? Compare this to the operation count of
$\frac{8}{3}n^{3} + \BigO{n^{2}}$ required for the same task using LU-decomposition (see Example 5.5).

{\bf Solution:}

%%%%%% Problem 5 %%%%%%
\item The Cholesky algorithm given on page 116 has all those wretched loops as in the Gaussin Elimination
algorithm in its simplest form. In view of Section 5.4 and the program {\tt ainvb} we should be able to
achieve also the Cholesky decomposition effect more efficiently.

Write a code implementing the Cholesky decomposition with only one loop (on $k$), utilizing outer products.

{\bf Solution:}

%%%%%% Problem 6 %%%%%%
\item Consider the LU decomposition of an upper Hessenberg (no, it's not a place in Germany) matrix,
defined on the facing page, assuming that no pivoting is needed: $A = LU$.

\begin{enumerate}
\item Provide an efficient algorithm for this LU decomposition (do not worry about questions of memory access and vectorization).

{\bf Solution:}

\item What is the sparsity structure of the resulting matrix $L$ ({\em i.e.}, where are its non-zeros)?

{\bf Solution:}

\item How many operations (to a leading order) does it take to solve a linear system $A\mathbf{x} = \mathbf{b}$,
where $A$ is upper Hessenberg?

{\bf Solution:}

\item Suppose now that partial pivoting is applied. What are the sparsity patterns of the factors of $A$?

{\bf Solution:}
\end{enumerate}

\item For the arrow matrices of Example 5.15, determine the overall storage and flop count requirements
for solving the systems with $A$ and with $B$ in the general $n \times n$ case.

{\bf Solution:}

\end{enumerate}

%\begin{proof}
%Blah, blah, blah.  Here is an example of the \texttt{align} environment:
%Note 1: The * tells LaTeX not to number the lines.  If you remove the *, be sure to remove it below, too.
%Note 2: Inside the align environment, you do not want to use $-signs.  The reason for this is that this is already a math environment. This is why we have to include \text{} around any text inside the align environment.
%\begin{align*}
%\sum_{i=1}^{k+1}i & = \left(\sum_{i=1}^{k}i\right) +(k+1)\\
%& = \frac{k(k+1)}{2}+k+1 & (\text{by inductive hypothesis})\\
%& = \frac{k(k+1)+2(k+1)}{2}\\
%& = \frac{(k+1)(k+2)}{2}\\
%& = \frac{(k+1)((k+1)+1)}{2}.
%\end{align*}
%\end{proof}

\end{document}
