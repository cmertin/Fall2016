\documentclass[12pt]{article}

\usepackage[margin=1in]{geometry}
\usepackage{amsmath,amsthm,amssymb}
\usepackage{float}
\usepackage{graphicx}
\usepackage{bbold}
\usepackage{algorithm}
\usepackage{algcompatible}
\usepackage{csquotes}
\usepackage{url}
\usepackage{enumerate}

\newcommand{\N}{\mathbb{N}}
\newcommand{\Z}{\mathbb{Z}}
\newcommand{\abs}[1]{\left| #1 \right|}
\newcommand{\norm}[1]{\left|\left| #1 \right|\right|}
\newcommand{\ceil}[1]{\left\lceil #1 \right\rceil}
\newcommand{\floor}[1]{\left\lfloor #1 \right\rfloor}
\newcommand{\pprime}{\prime \prime}
\newcommand{\BigO}[1]{\mathcal{O}\left( #1 \right)}
\newcommand{\proj}[2][]{\textit{proj}_{\vect{#1}}\vect{#2}}
\newcommand{\vect}{\mathbf}
\newcommand{\Id}{\mathbb{1}}
\newcommand{\inv}[1]{ #1^{-1}}
\newcommand{\minn}{\text{min}}
\newcommand{\maxx}{\text{max}}
\renewcommand{\P}[1]{\left( #1 \right)}
\newcommand{\diag}[1]{\text{diag}\P{#1}}
\newcommand{\grad}{\nabla}
\newcommand{\laplacian}{\nabla^{2}}

\newenvironment{theorem}[2][Theorem]{\begin{trivlist}
\item[\hskip \labelsep {\bfseries #1}\hskip \labelsep {\bfseries #2.}]}{\end{trivlist}}
\newenvironment{lemma}[2][Lemma]{\begin{trivlist}
\item[\hskip \labelsep {\bfseries #1}\hskip \labelsep {\bfseries #2.}]}{\end{trivlist}}
\newenvironment{exercise}[2][Exercise]{\begin{trivlist}
\item[\hskip \labelsep {\bfseries #1}\hskip \labelsep {\bfseries #2.}]}{\end{trivlist}}
\newenvironment{problem}[2][Problem]{\begin{trivlist}
\item[\hskip \labelsep {\bfseries #1}\hskip \labelsep {\bfseries #2.}]}{\end{trivlist}}
\newenvironment{question}[2][Question]{\begin{trivlist}
\item[\hskip \labelsep {\bfseries #1}\hskip \labelsep {\bfseries #2.}]}{\end{trivlist}}
\newenvironment{corollary}[2][Corollary]{\begin{trivlist}
\item[\hskip \labelsep {\bfseries #1}\hskip \labelsep {\bfseries #2.}]}{\end{trivlist}}

\renewcommand{\algorithmicrequire}{\textbf{Input:}}
\renewcommand{\algorithmicensure}{\textbf{Output:}}

\DeclareMathOperator{\sech}{sech}
\DeclareMathOperator{\csch}{csch}

\begin{document}

\title{CS6210: Homework 6}
\author{Christopher Mertin}
\date{December 8, 2016}
\maketitle

\begin{enumerate}
%%%%%% Problem 1 %%%%%%
\item 

\begin{enumerate}
\item Using an orthogonal polynomial basis, find the best least squares polynomial approximations, $q_{2}(t)$ of degree at most 2 and $q_{3}(t)$ of degree at most 3, to $f(t) = e^{-3t}$ over the interval $[0,3]$.

[Hint: For a polynomial $p(x)$ of degree $n$ and a scalar $a > 0$ we have $\int e^{-ax}p(x)\text{d}x = -\frac{e^{-ax}}{a}\ \left( \sum_{j=0}^{n} \frac{p^{(j)}(x)}{a^{j}}\right)$, where $p^{(j)}(x)$ is the $j^{th}$ derivative of $p(x)$. Alternatively, just use numerical quadrature, e.g., the {\sc Matlab} function {\tt quad}.]

{\bf Solution:}

In the general case, we can use Legendre Polynomials. Were first we can find $x$ with respect to $t$ as being

\begin{align*}
x &= \frac{2t-a-b}{b-a} = \frac{2t-0-3}{3-0} = \frac{2t-3}{3} = \frac{2}{3}t - 1\\
\intertext{Giving us our parameters to be}
\phi_{0} &= 1\\
\phi_{1} &= x = \frac{2}{3}t - 1\\
\phi_{2} &= \frac{1}{2}\left( 3x^{2} - 1\right) = \frac{1}{2}\left( 3 \left(\frac{2}{3}t - 1\right)^{2}-1\right) = \frac{2}{3}t^{2} - 2t + 1\\
\phi_{3} &= \frac{1}{27}\left( 20t^{3} - 90t^{2} + 114t - 36\right)\\
c_{0} &= \frac{\int_{0}^{3}e^{-3t}\text{d}t}{\int_{0}^{3}\text{d}t} = \frac{1 - e^{-9}}{9}\\
c_{1} &= \frac{\int_{0}^{3}\left( \frac{2}{3}t - 1\right) e^{-3t}\text{d}t}{\int_{0}^{3}\left( \frac{2}{3}t - 1\right)^{2}\text{d}t} = \frac{-7 -11e^{-9}}{27}\\
c_{2} &= \frac{\int_{0}^{3}\left( \frac{2}{3}t^{2} - 2t + 1\right)e^{-3t}\text{d}t}{\int_{0}^{3}\left( \frac{2}{3}t^{2} - 2t + 1 \right)^{2}\text{d}t} = \frac{65 - 245 e^{-9}}{243}\\
c_{3} &= \frac{\int_{0}^{3}\left( \frac{1}{27}\left( 20t^{3} - 90t^{2} + 114t - 36\right)\right)e^{-3t}\text{d}t}{\int_{0}^{3}\left( \frac{1}{27}\left( 20t^{3} - 90t^{2} + 114t - 36\right)\right)^{2}\text{d}t} = \frac{-413 - 5299e^{-9}}{2187}
\end{align*}

We can then put $\phi_{i}$ and $c_{i}$ into our approximation of the formua:

\[
v(x) = \sum_{j=0}^{n}c_{j}\phi_{j}(x)
\]

By using $n = 2$ we get $q_{2}(x)$ and by using $n = 3$ we get $q_{3}(x)$.

\item Plot the error functions $f(t) - q_{2}(t)$ and $f(t) - q_{3}(t)$ on the same graph on the interval $[0,3]$. Compare the errors of the two approximating polynomials. In the least squares sense, which polynomial provides the better approximation?

[Hint: In each case you may compute the {\em norm} of the error, $\left( \int_{a}^{b}(f(t) - q_{n}(t))^{2}\text{d}t \right)^{2}$, using the {\sc Matlab} function {\tt quad}]

{\bf Solution:}

\begin{figure}[H]
\centering
\includegraphics[width=.75\textwidth]{plot1.pdf}
\caption{See {\tt prob1.py}}
\end{figure}

\item Without any computation, prove that $q_{3}(t)$ generally provides a least squares fit, which is never worse than with $q_{2}(t)$. 

{\bf Solution:}

In general, for lower power polynomial expressions the higher polynomial that is used, the more accurate the result. Therefore $q_{3}(t)$ would provide the better approximation since it is $n = 3$ compared to $n = 2$ for $q_{2}(x)$, and the powers are not high enough to produce high oscillations. 
\end{enumerate}

\item Let $f(x)$ be a given function that can be evaluated at points $x_{0} \pm jh,\ j=\{0,1,2,\ldots\}$ for any fixed value of $h$, $0 < h \ll 1$.

\begin{enumerate}
\item Find a second order formula (i.e., trunctation error $\BigO{h^{2}}$) approximating the third derivative $f^{\prime \prime \prime}(x_{0})$. Give the formula, as well as an expression for the truncation error, i.e. not just its order

{\bf Solution:}

We can use the following four equations:

\begin{align*}
f(x_{0}+h) &= f(x_{0}) + hf^{\prime}(x_{0}) + \frac{h^{2}}{2}f^{\pprime}(x_0) + \frac{h^{3}}{6}f^{(3)}(x_{0}) + \frac{h^{4}}{24}f^{(4)}(x_{0}) + \frac{h^{5}}{120}f^{(5)}(\xi_{1})\\
f(x_{0}-h) &= f(x_{0}) - hf^{\prime}(x_{0}) + \frac{h^{2}}{2}f^{\pprime}(x_0) - \frac{h^{3}}{6}f^{(3)}(x_{0}) + \frac{h^{4}}{24}f^{(4)}(x_{0}) - \frac{h^{5}}{120}f^{(5)}(\xi_{1})\\
f(x_{0}+2h) &= f(x_{0}) + 2hf^{\prime}(x_{0}) + 2h^{2}f^{\pprime}(x_0) + \frac{8h^{3}}{6}f^{(3)}(x_{0}) + \frac{16h^{4}}{24}f^{(4)}(x_{0}) + \frac{32h^{5}}{120}f^{(5)}(\xi_{1})\\
f(x_{0}-2h) &= f(x_{0}) - 2hf^{\prime}(x_{0}) + 2h^{2}f^{\pprime}(x_0) - \frac{8h^{3}}{6}f^{(3)}(x_{0}) + \frac{16h^{4}}{24}f^{(4)}(x_{0}) - \frac{32h^{5}}{120}f^{(5)}(\xi_{1})\\
\end{align*}

After subtracting the first two equations and using the intermediate value theorem:

\begin{align*}
f(x_{0} + h) - f(x_{0} - h) &= 2hf^{\prime}(x_{0}) + \frac{h^{3}}{3}f^{(3)}(x_{0}) + \frac{h^{5}}{60}f^{(5)}(\eta_{1})
\end{align*}

Finding $f^{\prime}(x_{0})$

\begin{align*}
2hf^{\prime}(x_{0}) &= f(x_{0} + h) - f(x_{0} - h) - \frac{h^{3}}{3}f^{(3)}(x_{0}) - \frac{h^{5}}{60}f^{(5)}(\eta_{1})\\
f^{\prime}(x_{0}) &= \frac{1}{2h}\left( f(x_{0} + h) - f(x_{0} - h) - \frac{h^{3}}{3}f^{(3)}(x_{0}) - \frac{h^{5}}{60}f^{(5)}(\eta_{1})\right)\\
\intertext{After subtracting the second two equations and using the intermediate value theorem for the error terms:}
f(x_{0} + 2h) -& f(x_{0} - 2h) = 4hf^{\prime}(x_{0}) + \frac{8h^{3}}{3}f^{(3)}(x_{0}) + \frac{32h^{5}}{60}f^{(5)}(\eta_{2})\\
\intertext{Finding $f^{\prime}(x_{0})$:}
4hf^{\prime}(x_{0}) &= f(x_{0} + 2h) - f(x_{0} - 2h) - \frac{8h^{3}}{3}f^{(3)}(x_{0}) - \frac{32h^{5}}{60}f^{(5)}(\eta_{2})\\
f^{\prime}(x_{0}) &= \frac{1}{4h}\left( f(x_{0} + 2h) - f(x_{0} - 2h) - \frac{8h^{3}}{3}f^{(3)}(x_{0}) - \frac{32h^{5}}{60}f^{(5)}(\eta_{2})\right)
\end{align*}

After subtracting these two equations for $f^{\prime}(x_{0})$ and solving for $f^{(3)}(x_{0})$, we have

\begin{align*}
\frac{6h^{3}}{2}f^{(3)}(x_{0}) &= \frac{1}{4h}\left( f(x_{0} + 2h) - f(x_{0} - 2h) + 2f(x_{0} - h) - 2f(x_{0} + h) - \frac{1}{2}h^{5}f^{(5)}(\eta)\right)\\
f^{(3)}(x_{0}) &=\frac{1}{8h^{3}}\left( f(x_{0} + 2h) - f(x_{0} - 2h) + 2f(x_{0} - h) - 2f(x_{0} + h) - \frac{1}{2}h^{5}f^{(5)}(\eta)\right)
\end{align*}

We used the intermediate value theorem to find $\eta$. 

\item Use your formula to provide approximations to $f^{(3)}(0)$ for the function $f(x) = e^{x}$ employing values $h = \{ 10^{-1}, 10^{-2}, \ldots, 10^{-9}\}$, with the default ${\tt Matlab}$ arithmetic. Verify that for the larger values of $h$ your formula is indeed second order accurate. Which value of $h$ gives the closest approximation to $e^{0} = 1$?

{\bf Solution:}

\begin{table}[H]
\centering
\begin{tabular}{l r}
\hline \hline
$h$ & $f^{\prime}(x_{0})$\\
\hline
0.1 & 2.50625625351e-07\\
0.01 & 2.50006250091e-13\\
0.001 & 2.50000076196e-19\\
0.0001 & 2.50022225146e-25\\
1e-05 & 2.77555756156e-31\\
1e-06 & 0.0\\
1e-07 & 5.55111512313e-38\\
1e-08 & 0.0\\
1e-09 & 0.0\\
\hline
\end{tabular}
\caption{See {\tt prob2.py}}
\end{table}

\item For the formual that you derived in (a), how does the roundoff error behave as a function of $h$, as $h\rightarrow 0$.

{\bf Solution:}

$\widetilde{f}(x)$ is the approximation for $f(x)$:

\begin{align*}
\widetilde{f}(x) + f(x) + e_{r}(x)\\
\intertext{We do the same calculations on page 422}
\abs{f^{\prime}(x_{0}) - \widetilde{D}} &= \abs{ (f^{\prime}(x_{0}) - D) + (D - \widetilde{D})} \leq \abs{ f^{\prime}(x_{0}) - D} + \abs{D - \widetilde{D}}\\
\intertext{As we calculated in part a}
\abs{f^{\prime}(x_{0}) - D} &= \frac{1}{16}h^{2}f^{(5)}(\eta)\\
\intertext{And if $M$ is a maximum for $f^{(5)}(x)$ on its whole domain, then}
\abs{f^{\prime}(x_{0}) - D} &\leq \frac{1}{16}h^{2}M\\
\intertext{As an upper bound for $\abs{D - \widetilde{D}}$ we have $6\epsilon/h^{3}$ because: two $\epsilon$ for points $x_{0} + h$ and $x_{0} - h$ and two $2\epsilon$ for $x_{0} + 2h$ and $x_{0} - 2h$}
\abs{D - \widetilde{D}} &= \frac{6\epsilon}{h^{3}}\\
\end{align*}

Puttin these two together

\[
\abs{f^{\prime}(x_{0}) - \widetilde{D}} = \abs{(f^{\prime}(x_{0}) - D) + (D - \widetilde{D})} \leq \abs{f^{\prime}(x_{0}) - D} + \abs{D - \widetilde{D}} \leq \frac{1}{16}h^{2}M + \frac{6\epsilon}{h^{3}}
\]

\item How would you go about obtaining a forth order formula for $f^{(3)}(x_{0})$ in general? (You don't have to actually derive it: just describe in one or two sentences.) How many points would this formula require?

{\bf Solution:}

We can use seven points: $x_{0}$, $x_{0} + h$, $x_{0} - h$, $x_{0} + 2h$, $x_{0} - 2h$, $x_{0} + 3h$, $x_{0} - 3h$, and do similar computations for the fourth order as we did above.
\end{enumerate}

\item Consider the derivation of an approximate formula for the second derivative $f^{\prime \prime}(x_{0})$ of a smooth function $f(x)$ using three points $x_{-1},\ x_{0}=x_{-1} + h_{0},\ \text{and}\ x_{1}=x_{0}+h_{1}$, where $h_{0} \neq h_{1}$. 

Consider the following two methods:

\begin{enumerate}[i.]
\item Define $g(x) = f^{\prime}(x)$ and seek a {\em staggered mesh}, centered approximiations as follows:

\begin{align*}
g_{1/2} &= \frac{f(x_{1}) - f(x_{0})}{h_{1}};\quad g_{-1/2} = \frac{f(x_{0}) - f(x_{-1})}{h_{0}}\\
f^{\prime \prime}(x_{0}) &\approx \frac{g_{1/2} - g_{-1/2}}{(h_{0} + h_{1})/2}
\end{align*}

The idea is that all of the differences are short(i.e., not long differences) and centered.

\item Using the second degree interpolating polynomial in Newton form, differentiated twice, define

\[
f^{\prime \prime}(x_{0}) \approx 2f[x_{-1}, x_{0}, x_{1}]
\]

\end{enumerate}

Here is where you come in:

\begin{enumerate}
\item Show that the above two methods are one in the same

{\bf Solution:}

\begin{align*}
f^{\pprime}(x_{0}) &= \frac{g_{1/2} - g_{-1/2}}{(h_{0} + h_{1})/2} = \frac{\frac{f(x_{1}) - f(x_{0})}{x_{1} - x_{0}} - \frac{f(x_{0}) - f(x_{-1})}{x_{0} - x_{-1}}}{\frac{x_{1} - x_{-1}}{2}}\\
&= \frac{\frac{f(x_{1}) - f(x_{0})}{x_{1} - x_{0}} - \frac{f(x_{0}) - f(x_{-1})}{x_{0} - x_{-1}}}{x_{1} - x_{-1}} = 2f[x_{-1}, x_{0}, x_{1}]
\end{align*}

\item Show that this method is only first order accurate in general

{\bf Solution:}

We can use the Taylor expansion of $f$ around $x_{1}$:

\begin{align*}
f(x_{1}) &= f(x_{0} + h_{1}) = f(x_{0}) + h_{1}f^{\prime}(x_{0}) + \frac{h_{1}^{2}}{2}f^{\pprime}(x_{0}) + \frac{h_{1}^{3}}{3!}f^{(3)}(\eta_{1})\\
\intertext{Now we can rewrite $g_{1/2}$ as}
g_{1/2} &= \frac{f(x_{1}) - f(x_{0})}{h_{1}} = f^{\prime}(x_{0}) + \frac{h_{1}}{2}f^{\pprime}(x_{0}) + \frac{h_{1}^{2}}{3!}f^{(3)}(\eta_{1})
\end{align*}

We can use the Taylor expansion of $f$ around $x_{0}$:

\begin{align*}
f(x_{0}) &= f(x_{-1} + h_{0}) = f(x_{-1}) + h_{0}f^{\prime}(x_{-1}) + \frac{h_{0}^{2}}{2}f^{\pprime}(x_{-1}) + \frac{h_{0}^{3}}{3!}f^{(3)}(\eta_{2})\\
\intertext{Now we can rewrite $g_{-1/2}$ as}
g_{-1/2} &= \frac{f(x_{0}) - f(x_{-1})}{h_{0}} = f^{\prime}(x_{-1}) + \frac{h_{0}}{2}f^{\pprime}(x_{-1}) + \frac{h_{0}^{2}}{3!}f^{(3)}(\eta_{2})
\end{align*}

To prove that $f^{\pprime}(x_{0})$ is first order accurate we just need to consider the last expression in $g_{1/2}$ and $g_{-1/2}$ in the formula $f^{\pprime}(x_{0})$. The formula for $f^{\pprime}(x_{0})$ is:

\[
f^{\pprime}(x_{0}) = \frac{g_{1/2} - g_{-1/2}}{(h_{0} + h_{1})/2}
\]

So we just consider the last expression in $g_{1/2}$ and $g_{-1/2}$:

\[
f^{\pprime}(x_{0}) = 2\frac{\frac{h_{1}^{2}}{3!}f^{(3)}(\eta_{1}) - \frac{h_{0}^{2}}{3!}f^{(3)}(\eta_{2})}{h_{0} + h_{1}}
\]

we find $\eta$ by using the intermediate value theorem:

\begin{align*}
f^{\pprime} &=  2\frac{\frac{h_{1}^{2}}{3!}f^{(3)}(\eta_{1}) - \frac{h_{0}^{2}}{3!}f^{(3)}(\eta_{2})}{h_{0} + h_{1}}\\
                    &= 2 \frac{\left(\frac{h_{1}^{2}}{3!} - \frac{h_{0}^{2}}{3!}\right)f^{(3)}(\eta)}{h_{0} + h_{1}}
\end{align*}

The numerator is of $\BigO{h^{2}}$ while the denominator is $\BigO{h}$, so it's overall $\BigO{h}$.

\item Run the two methods for the example depicted in Table 14.2 (but for the second derivative of $f(x) = e^{x}$). Report your findings.

{\bf Solution:}



\end{enumerate}

\item Continuing with the notation of Exercise 12 (page 437), one could define 

\[
	g_{1/2} = \frac{f_{1} - f_{0}}{h}\quad \text{and}\quad g_{-1/2}=\frac{f_{0} - f_{-1}}{h}
\]

These approximate to second order the first derivative values $f^{\prime}(x_{0} + h/2)$ and $f^{\prime}(x_{0} - h/2)$, respectively. Then define

\[
	f_{pp_{0}} = \frac{g_{1/2} - g_{-1/2}}{h}
\]

All three derivative approximations here are centered (hence second order), and they are applied to first derivatives and hence have roundoff error increeasing proportionally to $h^{-1}$, not $h^{-2}$. Can we manage to (partially) cheat the hangman way?!

\begin{enumerate}
\item Show that in exact arithmetic $f_{pp_{0}}$ defined above and in Exercise 12 are one in the same

{\bf Solution:}

\begin{align*}
pp_{0} &= \frac{g_{1/2} - g_{-1/2}}{h} = \frac{\frac{f_{1} - f_{0}}{h} - \frac{f_{0} - f_{-1}}{h}}{h}\\
             &= \frac{f_{1} - f_{0} - f_{0} + f_{-1}}{h^{2}}
\end{align*}

Which is equal to the expression in Exercise 12.

\item Implement this method and compare to the results of Exercise 12. Explain your observations

{\bf Solution:}

\begin{table}[H]
\centering
\begin{tabular}{l r}
\hline \hline
$j$ & $f_{pp_{0}}$\\
\hline
1 & -0.85691243732 \\
2 & -0.924297937378 \\
3 & -0.931262645582 \\
4 & -0.931961418632 \\
5 & -0.932031319001 \\
6 & -0.932038309265 \\
7 & -0.932039008195 \\
8 & -0.932039077028 \\
9 & -0.932039079249 \\
10 & -0.932039001533 \\
11 & -0.932037780288 \\
12 & -0.932032229186 \\
13 & -0.93192120687 \\
14 & -0.931477117702 \\
15 & -0.921485110439 \\
16 & -0.777156116494 \\
17 & 0.0 \\
\hline
\end{tabular}
\caption{See {\tt prob4.py}}
\end{table}

\end{enumerate}

\item Consider the numerical differentiation of the function $f(x) = c(x) e^{x/\pi}$ defined on $[0, \pi]$, where 

\[
	c(x) = j, \quad \frac{1}{4}(j-1)\pi \leq x < \frac{1}{4}j\pi, \quad j = \{ 1, 2, 3, 4\}
\]

\begin{enumerate}
\item Contemplating a difference approximation with step size $h = \pi/n$ [fixed from errata], explain why it is a very good idea to ensure that $n$ is an integer multiple of $4,\ n = 4l$. 

{\bf Solution:}

We can simply show what happens at one boundary point, and the same argument holds true for the other

\begin{align*}
\intertext{$j = 1$:}
0 \leq\ &x \leq \frac{1}{4}\pi\\
\intertext{$f(\pi/4) = e^{1/4}$}
\intertext{$j = 2$:}
\frac{1}{4}\pi \leq\ &x \leq \frac{2}{4}\pi
\end{align*}

Where the value at $f(\pi/4)$ is $2e^{1/4}$, so the boundary points are discontinuous, which is the reason we want to split the interval to subintervals in which the boundary points are where the function is discontinuous.

\item With $n = 4l$, show that the expression $h^{-1}c(t_{i})\left( e^{x_{i+1}/\pi} - e^{x_{i}/\pi}\right)$ provides a second order approximation (i.e., $\BigO{h^{2}}$ error) of $f^{\prime}(t_{i})$, where $t_{i} = x_{i} + h/2 = (i + 1/2)h, i = \{ 0, 1, \ldots, n-1\}$

{\bf Solution:}

In order for the expression to be second order approximate, we need to find an approximation of $f^{\prime}(x)$ of the form

\begin{align*}
f^{\prime}(t_{i}) &= c(t_{i})\frac{e^{\left(t_{i} + \frac{h}{2}\right)/\pi} - e^{\left(t_{i} - \frac{h}{2}\right)/\pi}}{h} - \frac{h^{2}}{A}f^{B}(\xi)\\
\intertext{For which, $A$ and $B$ are constants. We start with the Taylor Series about two points}
f\left( t_{i} + \frac{h}{2}\right) &= f(t_{i}) + \frac{h}{2}f^{\prime}(t_{i}) + \frac{h^{2}}{2!\cdot 4}f^{\pprime}(t_{i}) + \frac{h^{3}}{3!\cdot 8}f^{(3)}(\xi_{1})\\
f\left( t_{i} - \frac{h}{2}\right) &= f(t_{i}) - \frac{h}{2}f^{\prime}(t_{i}) + \frac{h^{2}}{2!\cdot 4}f^{\pprime}(t_{i}) - \frac{h^{3}}{3!\cdot 8}f^{(3)}(\xi_{2})\\
\intertext{In subtracting the two and solving for $f^{\prime}(t_{i})$ we get}
f^{\prime}(t_{i}) &= \frac{f\left( t_{i} + \frac{h}{2}\right) - f\left( t_{i} - \frac{h}{2}\right)}{h} - \frac{h^{2}}{3!\cdot 4}f^{(3)}(\xi)\\
\intertext{where we have from the posed quesiton}
x_{i} &= t_{i} - \frac{h}{2}\\
x_{i+1} &= t_{i} + \frac{h}{2}
\end{align*}

Using the above values for $x_{i}$ with our equation for $f^{\prime}(t_{i})$

\begin{align*}
f^{\prime}(t_{i}) &= \frac{f(x_{i+1}) - f(x_{i})}{h} - \frac{h^{2}}{3!\cdot 4}f^{(3)}(\xi)\\
                            &= \frac{c(x_{i+1})e^{x_{i+1}/\pi} - c(x_{i})e^{x_{i}/\pi} - f(x_{i})}{h} - \frac{h^{2}}{3!\cdot 4}f^{(3)}(\xi)
\end{align*}

As the interval is on $[x_{i}, x_{i+1}]$, the function is constant in that subinterval since $c(x_{i+1}) = c(x_{i}) = c(t_{i})$

\end{enumerate}

\item The basic trapezoidal rule for approximating $I_{f} = \int_{a}^{b}f(x)\text{d}x$ is based on linear interpolation of $f$ at $x_{0} = a$ and $x_{1} = b$. The Simpson rule is likewise based on quadratic polynomial interpolation. Consider now a cubic Hermite polynomial, interpolating both $f$ and its derivative $f^{\prime}$ at $a$ and $b$. The osculating interpolation formula gives

\[
	p_{3}(x) = f(a) + f^{\prime}(a)(x-a) + f[a,a,b,b](x-1)^{2}(x - b)
\]

and integrating this yields (after some algebra)

\[
	I_{f} \approx \int_{a}^{b} p_{3}(x)\text{d}x = \frac{b-a}{2}[f(a) + f(b)] + \frac{(b-a)^{2}}{12}[f^{\prime}(a) - f^{\prime}(b)]
\]

This formula is called the {\bf corrected trapezoidal rule}

\begin{enumerate}
\item Show that the error in the basic corrected trapezoidal rule can be estimated by

\[
	E(f) = \frac{f^{(4)}(\eta)}{720}(b-a)^{5}
\]

{\bf Solution:}

Page 320 from the text tells us that for the Hermite Cubic Interpolation we have $m_{0} = m_{1} = 1$. Using the formula on page 321, we get the repitition of the first interpolation point, $a$, is $m_{0} + 1 = 2$. The second interpolation point $b$ is $m_{1} + 1  = 2$. From page 443, the interpolation error is:

\begin{align*}
f(x) - p_{n}(x) &= f[x_{0}, x_{1}, \ldots, x_{n}, x] \prod_{i=0}^{n}(x - x_{i})\\
\intertext{We should use two $a$'s and two $b$'s as the interpolation points, giving}
E(f) &= \int_{a}^{b}f[a,a,b,b,x]\left((x-a)^{2}(x-b)^{2} \right)\text{d}x
\intertext{Since the integrand is always positive, we can use the intermediate value problem to find $\xi$ in $[a,b]$ such that}
E(f) &= \int_{a}^{b}f[a,a,b,b,\xi]\left((x-a)^{2}(x-b)^{2} \right)\text{d}x
\intertext{where $f$ can be taken out of the integral since it's now independent of $x$}
E(f) &= f[a,a,b,b,\xi]\int_{a}^{b}\left((x-a)^{2}(x-b)^{2} \right)\text{d}x
\end{align*}

Now we can separate the two and solve for their values, giving

\begin{align}
\int_{a}^{b}\left((x-a)^{2}(x-b)^{2} \right)\text{d}x &= 0 - 2\frac{(a - b)^{5}}{3(4)5}= \frac{4(b-a)^{5}}{5!}
\end{align}

And from page 312 we have

\begin{align}
f[a,a,b,b,\xi] &= \frac{f^{(4)}(\eta)}{4!}
\end{align}

for some $\eta$ in $[a,b]$. Putting Equations (1) and (2) together, we get

\begin{align*}
E(f) = \frac{f^{(4)}(\eta)}{4!} \cdot \frac{4(b-a)^{5}}{5!} = \frac{f^{(4)}(\eta)(a - b)^{5}}{720}
\end{align*}

\item Use the basic corrected trapezoidal rule to evaluate approximations for $\int_{0}^{1}e^{x}\text{d}x$ and $\int_{0.9}^{1}e^{x}\text{d}x$. Compare errors to those of Example 15.2. What are your observations?

{\bf Solution:}

\begin{align*}
I_{f} &= \frac{b-a}{2}(f(a) + f(b)) + \frac{(b-a)^{2}}{12}(f^{\prime}(a) - f^{\prime}(b))
\intertext{For $a=0$ and $b=1$}
I_{f} &= \frac{1}{2}(1 + e) + \frac{1}{12}(1-e) = 1.7160\\
\intertext{For $a = 0.9$ and $b = 1$}
I_{f} &= \frac{1}{2}\left( e^{0.9} + e\right) + \frac{1}{12}\left( e^{0.9} - e\right) = 2.5674
\intertext{According to Example 15.2, the actual value is given, so we can find the error by the corrected trapezoidal rule, for $a = 0$ and $b = 1$}
E(f) &= \abs{1.7183 - 1.7160} = 0.0023\\
\intertext{for $a = 0.9$ and $b = 1$}
E(f) &= \abs{1.7183 - 2.5674} = 0.8491
\end{align*}

In $[0,1]$, the corrected trapezoidal rule is more accurate than the regular trapezoidal rule and the midpoint rule, though it still underperforms when compared to Simpson's rule ($\xi = 0.0006$). However, in $[0.9, 1]$ all of the methods are more accurate than the corrected trapezoidal rule.

\end{enumerate}

\item 

\begin{enumerate}
\item Derive a formula for the {\em composite midpoint rule}. How many function evaluations are required?

{\bf Solution:}

We want to prove the following equation

\[
\int_{a}^{b}f(x)\text{d}x = h\sum_{i=1}^{r}f\left( a + (i - 1/2)h\right)
\]

with $h = (b-a)/r$. According to the formula on page 447, we have

\begin{align*}
\int_{a}^{b}f(x)\text{d}x &= \sum_{i=1}^{r}\int_{t_{i-1}}^{t_{i}}f(x)\text{d}x\\
\intertext{where $t_{i} = t_{i-1} + h$ since $[a,b]$ was divided evenly. Therefore, by induction we have $t_{i-1} = t_{i-2} + h = \ldots = a + (i-1)h$. From page 443, the value for the integral in the normal midpoint rule is:}
\int_{t_{i-1}}^{t_{i}} f(x)\text{d}x &= (t_{i} - t_{i-1})f\left( \frac{t_{i-1} + t_{i}}{2}\right) = hf\left( \frac{t_{i-1} + t_{i-1} + h}{2}\right)\\
                                                          &= hf(t_{i-1} + h/2) = hf(a + (i-1)h + h/2) = hf(a + (i-1 + 1/2)h)\\
                                                          &= hf(a + (i - 1/2)h)
\intertext{Finally, giving}
\int_{a}^{b}f(x)\text{d}x &= \sum_{i=1}^{r}\int_{t_{i-1}}^{t_{i}}f(x)\text{d}x = h\sum_{i=1}^{r}f(a + (i - 1/2)h)
\end{align*}

\item Obtain an expression for the error in the composite midpoint rule. Conclude that this method is second order accurate

{\bf Solution:}

From page 453, the upper bound should be of the form 

\[
\frac{f^{\pprime}(\xi)(b-a)h^{2}}{24}
\]

and from page 445, the error for the midpoint rule is

\[
\frac{f^{\pprime}(\xi)(b - a)^{3}}{24}
\]

Where we can find the error of the function as being

\begin{align*}
E(f) &= \sum_{i=1}^{r}f^{\pprime}(\xi_{i})\frac{(t_{i} - t_{i-1})^{3}}{24} = \sum_{i=1}^{r}f^{\pprime}(\xi_{i})\frac{h^{3}}{24}
\intertext{Where we can find $\xi$ for the entire domain $[a, b]$, instead of $\xi_{i}$ for each subinterval $[t_{i-1}, t_{i}]$, so:}
E(f) &= \sum_{i=1}^{r}f^{\pprime}(\xi)\frac{h^{3}}{24} = rf^{\pprime}(\xi)\frac{h^{3}}{24}\\
       &= f^{\pprime}(\xi)\frac{h^{3}}{24}\frac{b-a}{h} = f^{\pprime}(\xi)\frac{h^{2}(b-a)}{24}
\end{align*}

Which can be completed by adding the norm.

\end{enumerate}

\item Suppose that the interval of integration $[a, b]$ is divided into equal subintervals of length $h$ each such that $r = (b - a)/h$ is even. Denote by $R_{1}$ the result of applying the composite trapezoidal method with step size $2h$ and by $R_{2}$ the result of applying the same method with step size $h$. Show that one application of Richardson extrapolation, reading 

\[
S = \frac{4R_{2} - R_{1}}{3}
\]

yields the composite Simpson method

{\bf Solution:}

\begin{align*}
\intertext{$R_{2}:$ step size: $h$}
R_{2} : \int_{a}^{b}f(x)\text{d}x &= \frac{h}{2}\left( f(a) + 2\sum_{k=1}^{r-1} f(a + kh) + f(b)\right)\\
\intertext{Where we can split the summation into even and odd, giving}
&= \frac{h}{2}\left( f(a) + 2\sum_{k=1}^{r/2 - 1} f(a + 2kh) + 2\sum_{k=1}^{r/2-1}f(a + (2k - 1)h) + f(b) \right)
\end{align*}

\begin{align*}
\intertext{$R_{1}:$ step size: $2h$. With it being $2h$, the number of subintervals is $r/2$}
R_{1}: \int_{a}^{b}f(x)\text{d}x &= \frac{2h}{2}\left( f(a) + 2\sum_{k=1}^{r/2-1} f(a + 2kh) + f(b) \right)\\
\end{align*}

leading to

\begin{align*}
4R_{2} - R_{1} &= 2hf(a) + 4h\sum_{k=1}^{r/2-1} f(t_{2k-1}) + 4h\sum_{k=1}^{r/2-1}f(t_{2k}) + 2hf(b) - hf(a)\ldots \\
                          &\ldots - 2h\sum_{k=1}^{r/2-1}f(t_{2k}) - hf(b)\\
                          &= hf(a) + 4h\sum_{k=1}^{r/2-1}f(t_{2k-1}) + 2h\sum_{k=1}^{r/2-1} f(t_{2k}) + hf(b)
\end{align*}

Where if we divide the above by 3, it completes the proof.

\item Using Romberg integration, compute $\pi$ to 8 digits (i.e. 3.xxxxxxxx) by obtaining approximations to the integral

\[
\pi = \int_{0}^{1}\left( \frac{4}{1 + x^{2}}\right)\text{d}x
\]

Describe your solution approach and provide the appropriate Romberg table.

Compare the computational effort (function evaluations) of Romberg integration to that using the adaptive routine developed in Section 15.4 with {\tt tol}$= 10^{-7}$.

You may find for some of the rows of your Romberg table that only the first step of extrapolation improves the approximation. Explain this phenomenon.

[Hint: Reconsider the assumed form of the composite trapezoidal method's truncation error and the effects of extrapolation for this particular integration]

{\bf Solution:}

\begin{table}[H]
\centering
\begin{tabular}{c c c c c c }
\hline\hline
3.00000000 &  &  &  &  & \\
3.10000000 & 3.13333333 &  &  &  & \\
3.13117647 & 3.14156863 & 3.14211765 &  &  & \\
3.13898849 & 3.14159250 & 3.14159409 & 3.14158578 &  & \\
3.14094161 & 3.14159265 & 3.14159266 & 3.14159264 & 3.14159267 & \\
3.14142989 & 3.14159265 & 3.14159265 & 3.14159265 & 3.14159265 & 3.14159265\\
\hline
\end{tabular}
\caption{Romberg Table, see {\tt prob9.py}}
\end{table}

Function Evaluations: 33

\end{enumerate}

%\begin{proof}
%Blah, blah, blah.  Here is an example of the \texttt{align} environment:
%Note 1: The * tells LaTeX not to number the lines.  If you remove the *, be sure to remove it below, too.
%Note 2: Inside the align environment, you do not want to use $-signs.  The reason for this is that this is already a math environment. This is why we have to include \text{} around any text inside the align environment.
%\begin{align*}
%\sum_{i=1}^{k+1}i & = \left(\sum_{i=1}^{k}i\right) +(k+1)\\
%& = \frac{k(k+1)}{2}+k+1 & (\text{by inductive hypothesis})\\
%& = \frac{k(k+1)+2(k+1)}{2}\\
%& = \frac{(k+1)(k+2)}{2}\\
%& = \frac{(k+1)((k+1)+1)}{2}.
%\end{align*}
%\end{proof}

\end{document}
